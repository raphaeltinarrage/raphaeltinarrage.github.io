%%%%%%%%%%%%%%%%%%%%%% Document class

\documentclass[11pt]{article}
\usepackage[a4paper, top=1.25in, bottom=1.25in, left=1.250 in, right=1.250in]{geometry}
\usepackage[portuguese]{babel}

%%%%%%%%%%%%%%%%%%%%%% Packages

\usepackage{amsmath}                         % basic math
\usepackage{amssymb}                         % math symbols
\usepackage[capitalise,nameinlink]{cleveref} % \cref, \Cref, \crefrange

%%%%%%%%%%%%%%%%%%%%%% Macros

\newcommand{\R}{\mathbb R}
\providecommand{\sin}{}\renewcommand{\sin}{\mathop{\rm sen}\nolimits}
\providecommand{\div}{}\renewcommand{\div}{\mathop{\rm div}\nolimits}
\DeclareMathOperator{\rot}{rot}
\newcommand{\HRule}{\rule{\linewidth}{0.5mm}} 

%%%%%%%%%%%%%%%%%%%%%%%%%%%%%%%%%%%%%%%%%%%%%%%%%%%%%%%%%%%%%%%%%%%
%
% Document
%
%%%%%%%%%%%%%%%%%%%%%%%%%%%%%%%%%%%%%%%%%%%%%%%%%%%%%%%%%%%%%%%%%%%

\begin{document}

\begin{center}
\Large{\textsc{FGV/EMAp, Matemática Aplicada, 3$^\text{o}$ período, 2024}}
\vspace{.25cm}

\HRule
\\\vspace{.25cm}

\LARGE{\textsc{Cálculo Vetorial --- Teste I}} 
\HRule
\end{center}
\vspace{.5cm}

\noindent\textbf{Modalidades}\textbf{.} 
Este teste consiste em dois problemas, cada questão valendo um ponto, em um total de dez pontos. As respostas devem ser argumentadas, e a aplicação de propriedades ou teoremas deve ser feita mediante a verificação das hipóteses. Os alunos têm a opção de devolver suas respostas por escrito em papel, pessoalmente ou por foto para o e-mail \texttt{raphael.tinarrage@fgv.br}, ou como um documento escrito com um software de redação matemática, enviado para o mesmo e-mail. 
Este documento será entregue aos alunos no 13/03/2024, e o prazo final é o 27/03/2024 às 11h10.
\vspace{.75cm}

\noindent\textbf{Problema 1} (campos vetoriais fechados no toro)\textbf{.} 
Sejam os conjuntos
\begin{align*}
L &= \{(x,y,z)\in\R^3 \mid x=0,~y=0\},\\
S &= \{(x,y,z)\in\R^3 \mid x^2+y^2=1,~z=0\},\\
\Omega &= \R^3 \setminus \big(L \cup S)
\end{align*}
e os campos vetoriais
\vspace{-0.25cm}
\begin{center}
\begin{minipage}[t]{0.3\linewidth}
\begin{align*}
F\colon\Omega&\longrightarrow\R^3\\
\begin{pmatrix} x \\ y \\ z\end{pmatrix}
&\longmapsto
\begin{pmatrix} 
\dfrac{-y}{x^2+y^2}
\vspace{.25cm}\\
\dfrac{x}{x^2+y^2}
\vspace{.25cm}\\ 
0
\end{pmatrix}
\end{align*}
\end{minipage}
\begin{minipage}[t]{0.68\linewidth}
\begin{align*}
G\colon\Omega&\longrightarrow\R^3\\
\begin{pmatrix} x \\ y \\ z\end{pmatrix}
&\longmapsto
\begin{pmatrix} 
\dfrac{-xz}{\sqrt{x^2+y^2}\bigg(\big(\sqrt{x^2+y^2}-1\big)^2+z^2\bigg)}
\vspace{.25cm}\\
\dfrac{-yz}{\sqrt{x^2+y^2}\bigg(\big(\sqrt{x^2+y^2}-1\big)^2+z^2\bigg)}
\vspace{.25cm}\\ 
\dfrac{\sqrt{x^2+y^2}-1}{\big(\sqrt{x^2+y^2}-1\big)^2+z^2}
\end{pmatrix}
\end{align*}
\end{minipage}
\end{center}

\begin{enumerate}
\item[\textbf{(a)}] Justifique que $F$ e $G$ são bem definidos e são de classe $C^\infty$.
\item[\textbf{(b)}] Mostre que o rotacional destes campos vale zero. 
\item[\textbf{(c)}] Da pergunta anterior, podemos concluir que os campos são conservativos? Detalhe.
\item[\textbf{(d)}] Esboce $F$ no plano $\{(x,y,z)\in\R^3\mid z=0\}$ e $G$ no plano $\{(x,y,z)\in\R^3\mid y=0\}$.
\item[\textbf{(e)}] Mostre que as duas seguintes curvas são curvas integrais de $F$ e $G$, respectivamente. Conclua que os campos não são conservativos.
\vspace{-0.5cm}
\begin{center}
\begin{minipage}{0.49\linewidth}
\begin{align*}
\gamma_F\colon&\R\longrightarrow\R^3\\
t&\longmapsto \begin{pmatrix}
\cos(4t)/2\\ \sin(4t)/2 \\ 0
\end{pmatrix}
\end{align*}
\end{minipage}
\begin{minipage}{0.49\linewidth}
\begin{align*}
\gamma_G\colon&\R\longrightarrow\R^3\\
t&\longmapsto \begin{pmatrix}
1+\cos(4t)/2\\ 0\\ \sin(4t)/2
\end{pmatrix}.
\end{align*}
\end{minipage}
\end{center}
\end{enumerate}

\noindent\underline{Indicações:} Um campo vetorial conservativo tem rotacional nulo. Além disso, uma curva integral para o campo $F\colon\R^3\rightarrow\R^3$ é uma curva $\gamma\colon\R\rightarrow\R^3$ tal que $\gamma' = F\circ \gamma$.
Se um campo admite uma curva integral não-constante e não-injetora, então não é conservativo.
\vspace{.10cm}

\noindent\underline{Comentário:} 
Como veremos mais adiante, a existência de campos vetoriais conservativos de rotacional nulo depende da topologia do espaço subjacente. 
Por exemplo, se o domínio for simplesmente conexo, então tais campos não existem (esse resultado é o lema de Poincaré). 
Por outro lado, o domínio $\Omega$ em nosso problema é, topologicamente falando, um toro (mais precisamente, $\Omega$ se retrai em um toro). 
No toro, há essencialmente dois campos vetoriais conservativos com rotacional nulo: são os campos $F$ e $G$ acima.
\vspace{1cm}

\noindent\textbf{Problema 2} (velocidade da luz)\textbf{.} 
Neste problema, trabalharemos no espaço $\R^4$, cujos pontos serão denotados genericamente $(x,y,z,t)$, sendo as três primeiras coordenadas entendidas como coordenadas espaciais e a quarta como uma coordenada temporal.
Dada uma função $F\colon\R^4\rightarrow\R^3$, que veremos como um campo vetorial dinâmico (ou seja, dependente do tempo), calcularemos o rotacional ($\rot$), o divergente ($\div$) e o laplaciano vetorial ($\nabla^2$) com relação às coordenadas espaciais.
Isto é, denotando $(F_1,F_2,F_3)$ as componentes de $F$, consideraremos
\begin{align*}
\rot F &= \left(
\frac{\partial F_3}{\partial y}-\frac{\partial F_2}{\partial z},~~
\frac{\partial F_1}{\partial z}-\frac{\partial F_3}{\partial x},~~ \frac{\partial F_2}{\partial x}-\frac{\partial F_1}{\partial y}
\right),\\
\div F &= \frac{\partial F_1}{\partial x} + \frac{\partial F_2}{\partial y} + \frac{\partial F_3}{\partial z},\\
\nabla^2 F &= 
\begin{pmatrix}
\dfrac{\partial^2 F_1}{\partial x^2}+\dfrac{\partial^2 F_1}{\partial y^2}+\dfrac{\partial F_1^2}{\partial z^2}\vspace{.2cm}
\\
\dfrac{\partial^2 F_2}{\partial x^2}+\dfrac{\partial^2 F_2}{\partial y^2}+\dfrac{\partial^2 F_2}{\partial z^2}\vspace{.2cm}
\\
\dfrac{\partial^2 F_3}{\partial x^2}+\dfrac{\partial^2 F_3}{\partial y^2}+\dfrac{\partial^2 F_3}{\partial z^2}
\end{pmatrix}.
\end{align*}
Este problema trata das equações de Maxwell no vácuo sem carga e sem corrente:
\begin{align}
\div E &= 0 \label{eq:testeI_maxwell_1}\\
\div B &= 0 \label{eq:testeI_maxwell_2}\\
\rot E &= -\dfrac{\partial B}{\partial t} \label{eq:testeI_maxwell_3}\\
\rot B &= \mu_0\epsilon_0 \dfrac{\partial E}{\partial t} \label{eq:testeI_maxwell_4}
\end{align}
onde $E,B\colon\R^4\rightarrow\R^3$ são funções de classe $C^2$, chamadas respectivamente de campo elétrico e campo magnético, $\mu_0 = 4\pi\times10^{-7}$ é a permeabilidade do vácuo e $\epsilon_0 \approx 8.85\times10^{-12}$ é a permissividade do vácuo.
Além disso, dado $\omega\in\R$, $k\in\R^3$ unitário e $n\in\R^3$ ortogonal a $k$, pomos a seguinte função, chamada de onda plana monocromática:
\begin{align}
F_{(\omega, k, n)}\colon \R^4&\longrightarrow\R^3 \label{eq:onda_plana}\\
(x,z,y,t)&\longmapsto \cos\big( \big\langle k, (x,y,z) \big\rangle - t\omega \big) n.\nonumber
\end{align}
O termo $\omega$ é chamado de velocidade, $k$ de vetor de propagação e $n$ de polarização.

\begin{enumerate}
\item[\textbf{(f)}] Sejam $\omega\in\R$ e $k,n\in\R^3$ ortogonais. Calcule $\rot F_{(\omega, k, n)}$, $\div F_{(\omega, k, n)}$ e $\frac{\partial}{\partial t} F_{(\omega, k, n)}$.
\item[\textbf{(g)}]
Seja $\omega\in\R$, $k\in\R^3$ unitário, $n\in\R^3$ ortogonal a $k$ , e define $\widetilde{n} = \frac{1}{\omega} k\times n$.
Além disso, consideremos as funções $E = F_{(\omega, k, n)}$ e $B = F_{(\omega, k, \widetilde{n})}$.
Mostre que $E$ e $B$ são soluções das equações \eqref{eq:testeI_maxwell_1}, \eqref{eq:testeI_maxwell_2}, \eqref{eq:testeI_maxwell_3} e \eqref{eq:testeI_maxwell_4} se e somente se $\omega = 1/\sqrt{\mu_0\epsilon_0}$.
\item[\textbf{(h)}] Seja $F\colon\R^4\rightarrow\R^3$ de classe $C^2$. Prove que $\frac{\partial}{\partial t} \rot F = \rot \frac{\partial}{\partial t} F$.
\item[\textbf{(i)}] 
Sejam $E,B\colon\R^4\rightarrow\R^3$ soluções de classe $C^2$ das equações \eqref{eq:testeI_maxwell_1}, \eqref{eq:testeI_maxwell_2}, \eqref{eq:testeI_maxwell_3} e \eqref{eq:testeI_maxwell_4}.
Mostre
\begin{align}
\nabla^2 E = \mu_0\epsilon_0 \dfrac{\partial^2 E}{\partial t^2}, \label{eq:testeI_ondas_1}\\
\nabla^2 B = \mu_0\epsilon_0 \dfrac{\partial^2 B}{\partial t^2}. \label{eq:testeI_ondas_2}
\end{align}
Poderemos aplicar o operador rotacional aos membros das equações \eqref{eq:testeI_maxwell_3} e \eqref{eq:testeI_maxwell_4} e usar a pergunta anterior e a indicação abaixo.
\item[\textbf{(j)}] 
Dada $F\colon\R^2\rightarrow\R$ de classe $C^2$, e denotando genericamente os pontos de $\R^2$ como $(x,t)$, considere equação
\begin{align}
\dfrac{\partial^2 F}{\partial t^2} = \mu_0\epsilon_0 \dfrac{\partial^2 F}{\partial x^2}. \label{eq:testeI_ondas}
\end{align}
Mostre que para toda função $f\colon\R\rightarrow\R$ de classe $C^2$ e todos $(k,\omega)\in\R^3\times\R$, a seguinte função é solução da \cref{eq:testeI_ondas}:
\begin{align*}
F\colon \R^2&\longrightarrow\R\\
(x,t)&\longmapsto f( kx - t\omega ).
\end{align*}
\end{enumerate}

\noindent\underline{Indicação:} 
Uma função $F\colon\R^3\rightarrow\R^3$ de classe $C^2$ satisfaz a relação
$$
\rot(\rot(F)) = \nabla \div F - \nabla^2 F.
$$

\noindent\underline{Comentário:} 
O resultado da questão \textbf{(g)} é uma das previsões notáveis das equações de Maxwell: para que uma onda plana seja uma solução, sua velocidade deve ser igual a $1/\sqrt{\mu_0\epsilon_0}\approx2,9\times10^{8}$, ou seja, a velocidade da luz. 
Hoje sabemos que a luz é uma onda eletromagnética, e este resultado pode parecer tautológico. 
No entanto, em uma época em que esse conhecimento ainda não estava solidificado, as equações de Maxwell permitiram argumentar a favor da luz como uma onda eletromagnética.
Além disso, vemos na questão \textbf{(i)} que as equações admitem, em nosso caso (no vácuo, sem carga e sem corrente), uma desacoplagem dos campos elétrico e magnético. Elas seguem então a equação da onda. 
Em geral, a equação da onda admite uma grande variedade de soluções, como vemos na questão \textbf{(j)}, no caso unidimensional em espaço.
No entanto, ao adicionar as equações \eqref{eq:testeI_maxwell_1} e \eqref{eq:testeI_maxwell_2} de Mawxell, que atuam como valor inicial, mostra-se que as soluções são, na realidade, apenas combinações lineares das ondas planas monocromáticas da equação \eqref{eq:onda_plana} (na verdade, precisamos tomar ``combinações lineares infinitas'', por meio da teoria das séries de Fourier).

\end{document}